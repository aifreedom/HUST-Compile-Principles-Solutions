\newcommand{\vor}{\ |\ }

\newcommand{\doctitle}{编译原理第一次作业}
\documentclass[oneside,a4paper]{article}

\usepackage{parskip}
\usepackage{subfigure}
% \usepackage{geometry}
\usepackage{amsmath}
\usepackage[dvipdfm]{graphicx} 
\usepackage{amsthm,amssymb}
\usepackage{tikz}
\usepackage{fontspec,xltxtra,xunicode}


\usepackage{fancyvrb}
% \usepackage{fancybox}
\usepackage{listings}
\lstset{numbers=left, 
  numberstyle=\scriptsize,
  frame=single,
  flexiblecolumns=false,
  language=,
  texcl=true,
  escapechar=\%,
  basicstyle=\ttfamily\small, 
  breaklines=true,
  extendedchars=true,
  showstringspaces=false,
  keywordstyle=\bfseries}

% \usepackage{indentfirst} 


\usepackage{pstricks} 
\usepackage[dvipdfm,a4paper,
hyperindex=true,
backref=section,
bookmarks=true,
bookmarksnumbered=true,
pdfpagemode=UseOutlines,
pdffitwindow=true,
linkbordercolor=white, % 链接边框设置为白色
citebordercolor=white, % 链接边框设置为白色
urlbordercolor=white]{hyperref}
          

% \DefineShortVerb{\|}

\usepackage{xeCJK}
\setCJKmainfont[BoldFont={Adobe Heiti Std}, ItalicFont={Adobe Kaiti Std}]{Adobe Song Std}
\setCJKmonofont{Adobe Fangsong Std}
\xeCJKsetcharclass{"0}{"2E7F}{0}
\xeCJKsetcharclass{"2E80}{"FFFF}{1}

\setmainfont[Mapping=tex-text]{Linux Libertine O}
\setsansfont[Mapping=tex-text]{Linux Biolinum O} 
\setmonofont{Courier 10 Pitch} 
\punctstyle{quanjiao} 


\renewcommand{\figurename}{图}
\renewcommand{\tablename}{表}
\renewcommand{\lstlistingname}{程序清单}

\newenvironment{solve}{%
  \settowidth{\leftskip}{\textit{解:}\ }%
  \makebox[0pt][r]{\textit{解:}\ }%
  \ignorespaces}{\qed\par\ignorespacesafterend}

\author{\textsc{Computer Science} 0813\\谢松 U200814454}
\title{\doctitle}
\hypersetup{
  pdftitle={\doctitle},
  pdfauthor={谢松},
  pdfsubject={\doctitle}}
\usepackage{booktabs}

\begin{document}


\maketitle

%%% Local Variables: 
%%% mode: latex
%%% TeX-master: t
%%% End: 


\section{Ex 1.3}

\begin{solve}
  \emph{解释程序}接受某个语言的程序并立即运行这个源程序. 它和编译程序
  的不同之处在于它并不将高级语言程序翻译成二进制程序代码, 而是每获取一
  条语句就分析执行一条语句.
\end{solve}

\section{Ex 1.4}

\begin{solve}
  \begin{enumerate}
  \item 语法分析
  \item 语义分析
  \item 代码生成
  \item 词法分析
  \end{enumerate}
\end{solve}

\section{Ex 3.2}

\begin{solve}
  $\mathrm{G}[N]$ 的语言是 $\{0, 1, 2, 3, 4, 5, 6, 7, 8, 9\}^*$
\end{solve}

\section{Ex 3.4}

\begin{solve}
  $\mathrm{L}(\mathrm{G}[Z]) = \{a^nb^n | a,b>0\}$
\end{solve}

\section{Ex 3.5}

\begin{solve}
  \begin{enumerate}
  \item 允许0打头
    \begin{align*}
      E &\rightarrow FE \vor{} T\\
      F &\rightarrow 0 \vor{} 1 \vor{} 2 \vor{} 3 \vor{} 4 \vor{} 5 \vor{} 6 \vor{} 7 \vor{} 8 \vor{} 9\\
      T &\rightarrow 0 \vor{} 2 \vor{} 4 \vor{} 6 \vor{} 8
    \end{align*}
    
  \item 不允许0打头
    \begin{align*}
      E & \rightarrow HST\\
      S & \rightarrow \varepsilon \vor{} DS \\
      T & \rightarrow 0 \vor{} 2 \vor{} 4 \vor{} 6 \vor{} 8\\
      D & \rightarrow 0 \vor{} H\\
      H & \rightarrow 1 \vor{} 2 \vor{} 3 \vor{} 4 \vor{} 5 \vor{} 6 \vor{} 7 \vor{} 8 \vor{} 9
    \end{align*}
  \end{enumerate}
\end{solve}

\section{Ex 3.8}
\begin{solve}
  该文法是二义的. $abc$是该文法的一个句子, 可以找到它的两个不同的语法
  树(parse tree), 如图~\ref{fig:abc-parse}所示.

  \begin{figure}[hb]
    \begin{center}
      \begin{minipage}[c]{0.5\textwidth}
        \centering
        \begin{tikzpicture}
          [level distance = 10mm]
          \node {$S$}
          child {node {$A$}
            child {node {$ab$}}
          }
          child {node {$c$}};
        \end{tikzpicture}
      \end{minipage}%
      \begin{minipage}[c]{0.5\textwidth}
        \centering
        \begin{tikzpicture}
          [level distance = 10mm]
          \node {$S$}
          child {node {$a$}}
          child {node {$B$}
            child {node {$bc$}}
          };
        \end{tikzpicture}
      \end{minipage}
      \caption{$abc$不同的分析树}
      \label{fig:abc-parse}
    \end{center}
  \end{figure}
\end{solve}

\section{Ex 3.9}
\begin{solve}
  \begin{figure}[tb]
    \centering
    \begin{tikzpicture}
      [level distance = 10mm,
      level 1/.style={sibling distance = 20mm},
      level 2/.style={sibling distance = 10mm}]
      \node {$S$}
      child {node {$S$}
        child {node {$S$}
          child {node {$a$}}
        }
        child {node {$S$}
          child {node {$a$}}
        }
        child {node {$+$}}
      }
      child {node {$S$}
        child {node {$a$}}
      }
      child {node {$*$}};
    \end{tikzpicture}
    \caption{$aa+a*$的parse tree}
    \label{fig:aa-parse}
  \end{figure}
  串$aa+a*$的推导如下
  \begin{align*}
    S &\Rightarrow SS*\\
    &\Rightarrow SS+a*\\
    &\Rightarrow aa+a*
  \end{align*}
  构造parse tree如图~\ref{fig:aa-parse}所示.
  
\end{solve}

\section{Ex 3.11}

\begin{solve}
  $E + T * F$的推导如下
  \[E \Rightarrow E+T \Rightarrow E+T*F\]

  $E + T * F$ 的短语为$E$, $T$, $F$, $T*F$, $E+T*F$.

  直接短语为$E$, $T$, $F$, $T*F$.

  句柄为$E$.
\end{solve}

\section{Ex 3.14}
\begin{solve}
  \begin{enumerate}
  \item $\{a^nb^na^mb^m|n, m \le 0\}$
    \begin{align*}
      E & \rightarrow FF\\
      F & \rightarrow \varepsilon \vor aFb
    \end{align*}

  \item $\{1^n0^m1^m0^n | n, m \le 0\}$
    \begin{align*}
      E & \rightarrow F \vor 1E0\\
      F & \rightarrow \varepsilon \vor 0F1
    \end{align*}
  \item $\{WaW^r|W\in\{0|a\}^*\}$
    \begin{align*}
      E & \rightarrow a \vor aEa \vor 0E0
    \end{align*}
  \end{enumerate}
\end{solve}

\section{Ex 3.16}
\begin{solve}
  \begin{enumerate}
  \item $\{a^n|n\le 0\}$
    \begin{align*}
      E & \rightarrow \varepsilon \vor aE
    \end{align*}
  \item $\{a^nb^m|n,m\le 1\}$
    \begin{align*}
      E & \rightarrow aF \vor aE\\
      F & \rightarrow bF \vor b
    \end{align*}
  \item $\{a^nb^mc^k | n, m, k \le 0\}$
    \begin{align*}
      S & \rightarrow T \vor aS\\
      T & \rightarrow U \vor bT\\
      U & \rightarrow \varepsilon \vor cU
    \end{align*}
  \end{enumerate}
\end{solve}

\section{Ex 3.17}

\begin{solve}
  等价.
\end{solve}


\end{document}

%%% Local Variables: 
%%% mode: latex
%%% TeX-master: t
%%% End: 


%%% Local Variables: 
%%% mode: latex
%%% TeX-master: t
%%% End: 
